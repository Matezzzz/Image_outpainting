%%% A template for a~simple PDF/A file like a~stand-alone abstract of the~thesis.

\documentclass[12pt]{report}

\usepackage[a4paper, hmargin=1in, vmargin=1in]{geometry}
\usepackage[a-2u]{pdfx}
\usepackage[utf8]{inputenc}
\usepackage[T1]{fontenc}
\usepackage{lmodern}
\usepackage{textcomp}

\begin{document}

%% Do not forget to edit abstract_cz.xmpdata.

Dokreslování obrázků je úloha z oblasti generativní umělé inteligence, jejímž cílem je co nejrealističtěji rozšířit obrázek. Tato práce se snaží o vytvoření algoritmu využívajícího strojové učení, který bude schopen dokreslovat obrázek oblohy pomocí několika nových postupů z oboru. Natrénujeme tři modely, tokenizer pro převod obrázků na tokeny a zpět, maskovaný generativní transformer (MaskGIT), který je schopen dokreslovat tokeny, a super sampler, který umí zvětšit výsledný obrázek a přidat do něj detaily. Všechny modely natrénujeme čistě na obrázcích oblohy. Poté navrhneme postup, který zkombinuje natrénované modely k dokreslování obrázků. Nakonec popíšeme výsledky každého z modelů i výsledného algoritmu. Náš přínos je hlavně dodání fungující, open-source implementace včetně natrénovaných modelů, která bude schopna řešit úlohu dokreslování obrázků oblohy.


\end{document}
