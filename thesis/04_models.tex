\chapter{Models} \label{models}


This chapter describes all models used, how they are trained, and how they are utilized in the~outpainting algorithm.

Our outpainting algorithm consists of three distinct models - a~VQVAE tokenizer, the~MaskGIT transformer model, and a~diffusion model super sampler. The~tokenizer consists of two parts, an~encoder that can convert images to a~latent space consisting of discrete tokens, and a~decoder, which attempts to reconstruct the~original image given tokens. The~MaskGIT model takes in an~image converted to tokens using the~tokenizer, with some tokens getting masked out, and tries to reconstruct the~original tokens. The~super sampler gets an~image on input, upscales it, and tries to fill in the~missing details. For clarity, all models are visualized in \figref{all_models}.


\figureimg{all_models}{All the~models we use. On top, we have the~tokenizer - it can convert images to discrete tokens, and tokens back to images. In the~middle, we have MaskGIT - when provided with an~image with some of its' tokens masked (marked as black in the~image), it can replace the~missing tokens with reasonable values. The~bottom row shows the~diffusion model. Provided with a~low-resolution copy of the~image, it can upscale it and add some details.}


During outpainting, we use the~models as follows - first, we get the~image we are trying to extend, pass it through the~tokenizer encoder, receiving a~set of tokens. After that, we repeatedly call the~MaskGIT model, generating feasible tokens surrounding the~original image, until a~specified size is reached. When we are done, we use the~tokenizer decoder to convert the~tokens back to an~image, which is then upscaled using the~super sampler to create the~final result. The~process is described in more detail later in the~next chapter.


\section{Tokenizer}

The tokenizer is responsible for taking in images and converting them to a~latent space consisting of discrete tokens. The~architecture is based on the~vector quantization variational autoencoder \citep{vqvae} and consists of three parts - an~encoder, the~vector quantization layer, and a~decoder. The~encoder takes images as input, processing them using multiple residual blocks and downscaling convolutions, and outputs a~downscaled image of embedding vectors in an~embedding space. The~vector quantization layer then takes the~vectors in embedding (or latent) space, and attempts to represent each of them using a~discrete token, then it converts the~tokens back to the~embedding space. The~decoder then uses more residual layers and transposed convolutions to attempt to reconstruct the~original image from the~vectors in the~embedding space. Our variant of the~VQVAE trains on images of size $128 \times 128$ and uses a~latent space of dimensions $32 \times 32$. We describe the~architecture and training process of all model parts in this section.

In the~description of both the~encoder and decoder, we use the~variations on the~residual block introduced in \figref{residual_block_tokenizer}.

\figureimg{residual_block_tokenizer}{Residual block configurations used in the~tokenizer. All convolutions have $3 \times 3$ kernels and linear activations. The~dashed block in default convolution is active only when the~input image has a~different number of filters than $f$, and uses a~$1 \times 1$ kernel.}

We use the~encoder architecture as shown in \figref{tokenizer_encoder}:

\figureimg{tokenizer_encoder}{Tokenizer encoder architecture. Using 2 downscaling blocks results in a~4-times downscale, and using the~vector quantization layer on the~output results in an~output of $32 \times 32$ tokens}


The vector quantization layer can be viewed as an~extension to the~embedding layer with added support for converting vectors to discrete tokens. The~inverse operation is implemented by simply taking the~closest vector in the~embedding matrix according to the~L2 metric and using the~associated token, while the~conversion back is the~same as in the~original embedding layer. During the~forward pass, we apply both operations in order - first the~conversion to tokens, then embedding them back. To be able to train the~network using backpropagation, we need to be able to compute the~gradient of the~loss relative to the~layer inputs when provided with the~gradient of the~outputs. Thankfully, albeit our layer is not differentiable, as both the~layer inputs and outputs share the~same embedding space, VQVAE proposes to copy the~gradient from output to input, propagating useful information and enabling the~model to be trained.


The decoder uses an~architecture similar to the~encoder, just using upscaling convolutions instead of downscaling ones, as presented in \figref{tokenizer_decoder}.

\figureimg{tokenizer_decoder}{Tokenizer decoder architecture. Again, two upscaling convolutions are used, upscaling the~tokens four-fold while attempting to reconstruct the~original colors.}


Using the~terminology from VQVAE ($x$ is the~model input, $z_e(x)$ is the~encoder output, $z_q(x)$ is the~vector quantization layer output), we train the~model using the~following loss:

$$L = K\log p(x|z_q(x)) + \alpha ||sg(z_e(x)) - e||_2^2 + \beta ||z_e(x) - sg(e)||_2^2 + \gamma c_t ||e_t - v_t||_2^2$$

The first three terms are from the~VQVAE article, their rationale is as follows:

\begin{itemize}
\item $K\log p(x|z_q(x))$ - The~reconstruction loss multiplied by a~constant, $K$. When we interpret the~model outputs as means of a~normal distribution for each pixel, this term can be derived to be equal to the~mean squared error between decoder outputs and encoder inputs, multiplied by a~constant, $K$.
\item $\alpha ||sg(z_e(x)) - e||_2^2$ - The~VQVAE embedding loss. Using the~stop gradient $sg$ operator, it changes the~codebook vectors in the~vector quantization layer to be closer to encoder outputs.
\item $\beta ||z_e(x) - sg(e)||_2^2$ - VQVAE commitment loss. Attempts to move encoder predictions closer to embedding vectors to increase training stability.
\end{itemize}

We set all three constants to the same values as the original VQVAE article, with $K$ being equal to the~inverse of the~variance in the~input data, $K=1/Var(x)$, $\alpha=1$ and $\beta=0.25$.

We found that the~original VQVAE tended to leave some vectors unused, missing out on potential generation quality. For this reason, we introduce the~following loss: $\gamma \sum_{t \in T} c_t ||e_t - v_t||_2^2$. Going over all VQVAE tokens $T$, $c_t$ is $1$ if the~token was used for embedding the~last batch, $0$ otherwise. $e_t$ is the~closest encoder result vector from the~last batch, and $v_t$ is the~token embedding in the~codebook. The~loss moves unused embedding vectors closer to encoder outputs in the~last step, increasing their chance of being used the~next time. We use $\gamma=0.01$.

In the~following sections, we use the~following terminology: Using the~tokenizer/encoder for conversion to tokens means applying the~encoder and the~first part of the~VQ layer. Analogously, using the~decoder to convert tokens to an~image means embedding the~tokens using VQ, and using the~decoder to get an~image.



\section{MaskGIT}

The masked generative transformer model \citep{maskgit} is responsible for doing outpainting on tokens - given an~image with half or a~quarter of its' tokens masked out, it can replace the~hidden tokens with ones that feasibly complete the~image. We base our model structure on the~original MaskGIT article and only slightly modify the~training process. We first describe the~model architecture, how we use it when generating masked tokens, and finally, how we train it.

The MaskGIT architecture is greatly inspired by BERT \citep{BERT} from the~area of NLP, just generalized for images. It takes an~array of tokens as an~input, some of which may be replaced with a~special symbol, \texttt{MASK\_TOKEN}, to signify that the~model should try to predict the~token in this place, and outputs, for each place, a~set of logits, each one corresponding to a~token from the~tokenizer codebook.

The exact architecture used is presented below in \figref{maskgit}.

\figureimg{maskgit}{MaskGIT architecture. The~token embedding layer contains trainable embeddings for each token, while the~positional embeddings contain one for each sequence position. The~token embedding of a~\texttt{MASK\_TOKEN} is $0$.}


We now continue by describing the~way new tokens are generated, the~process used is the~same as in the~original article, but we provide a~brief summary. In theory, the~model would be able to generate all tokens in a~single pass, but we can get samples of much better quality if we allow multiple generation steps. We use the~cosine mask scheduling function $\gamma(x) = cos(2*pi*x)$ from the~original article and set the~\texttt{decode\_steps} and \texttt{generation\_temperature} parameters, specifying how many times we call the~model when generating the~image and how creative should the~model be when guessing tokens respectively.  Starting with an~input token array where some tokens are equal to \texttt{MASK\_TOKEN}, we run the~following algorithm:

$N =$ number of unknown tokens in input tokens

For $t = 1$ to \texttt{decode\_steps}:
\begin{itemize}
\item Use the~MaskGIT model to predict masked token logits. If we already know a~token, we replace its logit with infinity.
\item For each position, compute softmax over all logits to obtain token probabilities. Sample one token at random for every position - because we set the~token logit to infinity, this will preserve the~input tokens.
\item Compute the~number of tokens that should be masked after this step, $K=\gamma(\frac{t}{decode\_steps}) * N$
\item We obtain confidence values by adding logits corresponding to the~sampled tokens and random values from the~Gumbel distribution with a~scale equal to \texttt{generation\_temperature}.
\item We take the~$K$ lowest confidence values and replace their values in sampled tokens with the~\texttt{MASK\_TOKEN}. Then, we set the~input tokens for the~next step to sampled tokens.
\end{itemize}

When the~cycle finishes, we are left with the~newly created tokens.

During outpainting, we generally know a~part of the~columns or rows in the~input token image and need to generate the~tokens adjacent to them. As this is the~only thing we will use the~model for, we incorporate it into the~training process, modifying the~original MaskGIT. We define a~\texttt{mask\_ratio} constant, defining the~percentage of rows/columns that will be masked. During outpainting, we either know the~unmasked rows/columns, or we know both, and the~only missing part is one corner. Based on this, we define training masks, 8 in total, describing all possible usages of the~model, shown in \figref{maskgit_training_masks}.

\figureimg{maskgit_training_masks}{MaskGIT training masks. White and black pixels represent known and masked tokens respectively.}


We modify the~training process to use our training masks as follows:

\begin{itemize}
\item We sample a~ratio $t \in [0, 1]$
\item Using the~cosine scheduling function from the~section above, we obtain a~\texttt{mask\_ratio} $= \gamma(t)$
\item We take an~image and keep all tokens that are marked as unmasked in the~training mask. We mask all other tokens with probability \texttt{mask\_ratio}.
\item We minimize the~negative log-likelihood of the~masked tokens.
\end{itemize}




\section{Diffusion model super sampler}

Because the~tokenizer tends to lose some details when decoding from tokens, we use an~upscaling model loosely based on \citep{diffusion_super_sampler} to add details to the~final image and upscale it. For this, we use a~diffusion model with a~cosine generation schedule, where the~image we try to predict is edges - values that should be added to the~blurry image to generate a~sharp version. We describe the~architecture of the~model here, and how it is trained.

The model inputs are the~blurred 512x512 images, the~ratio of the~noise component relative to the~signal and the~edges with a~noise component added. We train the~model to predict the~noise component in generated edges. The~architecture we use is based on U-Net \citep{u_net}, and it uses the~same residual blocks as the~tokenizer for encoding/decoding. At each resolution, we use residual blocks to process the~data, then a~downscaling block into more filters, until images of size 8x8. We then do the~inverse, using upscaling convolutions into fewer filters, concatenate with the~processed images at the~same level created during downscaling, use a~residual block on the~whole stack, and upscale again, until we reach the~target resolution of 512x512. We then predict the~final noise in the~image using a~linear layer. To illustrate the~architecture, we first present some building blocks used in \figref{upscaler_blocks}, then the~architecture itself in \figref{upscaler}.

\figureimg{upscaler_blocks}{Super sampler downscaling and upscaling blocks. The~upscaling block uses two input parameters, denoted by arrows of different colors.}

\figureimg{upscaler}{Super sampler architecture. Input image, noisy edges, and noise variance are all used as model inputs, predicted noise is the~only output. To make the~illustration more compact, the~model execution starts to the~right, but continues to the~left below, in the~direction of the~arrows. During processing, the~model downscales the~image 6 times, to a~resolution of $8 \times 8$, then scales it back up. For clarity, we show the~amounts of filters used in the~final model, even though this is a~modifiable parameter.}


For the~training procedure, we need to find the~target edges. by taking a~high-resolution image, downscaling it to the~tokenizer input size, passing it through the~tokenizer, and upscaling it again, obtaining a~blurry image. We get the~edges by subtracting the~blurry image from the~original, high-resolution one. With this, we can train the~model like a~generic diffusion - we mix the~target edges with a~random amount of noise according to a~random time in our diffusion schedule, and the~model attempts to approximate the~noise.

Upscaling tokenizer outputs can be done simply - we first upscale the~image using bicubic interpolation and use the~diffusion model iterative decoding to estimate the~edges in it. After it is done, we add the~edges to the~blurry image, and clamp all values between 0 and 1, obtaining a~sharp copy.
