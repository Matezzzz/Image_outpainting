\chapter{Outpainting} \label{outpainting}


We now describe how to perform outpainting using the trained models. We assume we start with an image of the same size as the tokenizer input and want to expand it beyond its borders. We allow specifying the \texttt{outpaint\_range} and \texttt{samples} parameters - the outpaint range defines how many outpainting steps the model should take, and the samples define the quality of decoding. We use the tokenizer to convert the input image to tokens, then we use the MaskGIT to perform outpainting on the created tokens, and finally, we use the super sampler to generate high-quality samples. We visualize the whole process in \figref{algorithm}, each step is described in more detail below.

\figureimg{algorithm}{An overview of the outpainting algorithm. MaskGIT, tokenizer and super sampler are called multiple times during the last three steps, to cover the whole outpainted image.}


After converting the input image we reserve space around it for the tokens we want to generate and set them all to \texttt{MASK\_TOKEN}. After that, we move around the perimeter of known tokens in the order from \figref{outpainting_order} and use MaskGIT to generate new tokens around it, until the whole space is filled.

\figureimg{outpainting_order}{Outpainting order. We start with known tokens in the middle (white) and unknown everywhere else (black). Gray tokens are the ones currently being outpainted, light gray are known tokens currently being fed into the MaskGIT to create new ones.}


When all tokens have been generated, we want to convert them back to an image. Because calling the tokenizer decoder directly on neighboring blocks of tokens can introduce artifacts (as we have no guarantee that the borders will be smooth), we instead use a sliding window, call the decoder with many overlapping squares of tokens, and average the results. The centers of the generating blocks are on a regular grid, and the *samples* parameter specifies how many more times we call the decoder, relative to neighboring blocks. This process produces multiple overlapping images - to compute the final image, we use a weighted averaging of the colors on a given position. Using weights equal to 1 everywhere can sometimes still produce visible edges in the final image in places where one block ends. To eliminate this, we use smaller weights for colors near the edge of their output image, while those at the center have large ones.

The previous step created an outpainted image, but due to averaging and data loss when using the tokenizer, the output is often blurry. We fix this with the last network, the super sampler, which can add details to the outpainted image and upscale it. We once again work on a grid, this time, we only introduce small overlaps to avoid inconsistencies between neighboring parts of the image. Then we gradually go over all grid positions and call the super sampler for each one. If we already know some part of the result, because it was computed before, we alter the diffusion model generation slightly - during each decoding step, when we separate the edges into noise and signal data, and estimate the image in the next step using the two, we replace the estimated signal values with the values we already know. After every position is done, we are left with the upscaled image.
